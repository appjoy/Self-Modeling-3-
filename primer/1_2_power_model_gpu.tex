\subsection{GPU Power Model is App Dependent}
\label{subsec:gpu}

Next, we show the GPU power model can also be highly dependent on app usage. 
The GPU gives an example of a component that cannot be exercised alone and we show
how TPMD models the GPU power incrementally over the CPU model.

%%% Things discussed in this section
%% 1. Explain GPU is complex in modern smartphones
%% 2. Explain the GPU frequencies and power states in modern smartphones
%% 3. Explain complete GPU model
%% 4. What did we do
%% 5. Explain results and observation
%%   5a. Explain per scenario dependent modeling
%%   5b. Explain the reasons for the observation

%The GPU of modern phones can run at various frequencies and transition among various power states.
\paragraph{Parameters of the GPU power model}
In Moto Z3, the GPU can run at 7 different frequencies and has three power states,
Active, Slumber (offline), and Aware.
When the GPU is woken out of the Slumber state, it enters the Aware state for a brief duration.
% \comment{Similarly, Nexus 6 can run at 5 different frequencies and three power states; Active, Init and Nap. Init corresponds to Aware and Nap corresponds to Slumber of Moto Z3}
%{
% How does does it stay in Aware?
%This the initialization phase of the GPU.
% for a fixed duration? 535 uS on average ( with a 250 uS difference between 435 uS (min) and 635 us (max))
% }
The Active state has two sub states, Active-busy when the GPU is computing and Active-idle
when the GPU is idle. After staying in Active-idle for a threshold interval, the GPU enters the Slumber state.
Since the GPU never enters the Slumber or Aware state when an app is running, 
the parameters of the GPU model include only the GPU power draw in Active-busy and Active-idle in each frequency, $p^g_{busy}(g_k), p^g_{idle}(g_k)$, as listed in Table~\ref{tab:parameters}.
We abbreviate Active-busy and Active-idle states simply
as GPU Busy and Idle states in the rest of the paper.

\if 0
Accordingly, we model the GPU power draw as follows:
\begin{equation}
    Power_{GPU} = \sum_{j}\sum_{i} u_{ij}*p_{ij}
    \end{equation}
where $p_{ij}$ is the power parameter for the $i^{th}$ GPU frequency for the $j^{th}$ power state (Active-busy, Active-idle, Slumber, Aware) and $u_{ij}$ is the corresponding utilization in that frequency and power state.
\fi

\paragraph{TPMD methodology}
% Similar to CPU, the GPU of modern smartphones have also vastly developed.
Initially, we tried to use a GPU benchmark, 3DMark~\cite{3DMark}, to drive the GPU usage into Busy and Idle states under each frequency.
However, we found that the GPU benchmark renders very different consecutive frames which
result in rapidly fluctuating GPU utilizations and power monitor readings.
This makes aligning the power monitor readings with the GPU Busy periods difficult
which is needed to extract the GPU Busy state power at each frequency.
%
% In power trace it was difficult to identify busy and idle traces for GPU benchmarks.This is because utilization is not constant as it continually generates new frames and can’t be paused.
% In apps the GPU utilization is constant because when no input is given to  the game app it generates the same frame again and again.
%
% we can only generate average GPU power coefficients for GPU busy and idle states.
% Average GPU coefficient depends on GPU utilization i.e  the time GPU is busy to total GPU running time.
%Average GPU power coefficients so derived is fixed and can not cater to two apps which may have different GPU utilization.
In contrast, we observe that real apps tend to render very similar consecutive frames in a short period of time which result relatively stable GPU utilization and stable GPU power draw, 
making it much easier to extract the power monitor reading corresponding to a given GPU utilization
(see Figure~\ref{fig:power_trace_candycrushturorial} later). 
%  the rendering task often finishes in less than the 16.7ms per-frame interval and hence the GPU will go through busy and idle states in every 16.7ms interval, making it much easier to manually identity both states. 
% \questionaj{Why do we have to identify manually? Triggers are not available in atrace?
% Due to both drift and scaling their is vast alignment problem in aligning each 16.7 ms interval.
% Alignment problem prevent us from directly using triggers from the event trace.}
Based on this observation, we directly used selected apps that use 
only the CPU and GPU to derive the GPU power models. 
In particular, we derive the CPU model first using the integer-arithmetic microbenchmark\footnote{We found using the CPU model derived using memory-intensive microbenchmark
resulted in negative GPU coefficients.}
and then use the difference 
between the measured phone power and model-estimated CPU power as the ground-truth for the GPU power draw
in GPU power model derivation.


% 5. Explain results and observation

\begin{table*}[tp]
{\footnotesize
    \centering
    \caption{Moto Z3 GPU power model (Busy and Idle power per frequency) with the CPU fixed at 1.056 GHz.}
    \vspace{-0.1in}
    \begin{tabular}{|p{11.5mm}|p{22mm}|c|c|c|c|c|c|c|c|c|c|c|c|c|c|}
    \hline
    App & Scenario & \multicolumn{14}{c|}{GPU Frequency} \\
    \cline{3-16}
     &  & \multicolumn{2}{c|}{257 MHz} & \multicolumn{2}{c|}{342 MHz} & \multicolumn{2}{c|}{414 MHz} & \multicolumn{2}{c|}{515 MHz} & \multicolumn{2}{c|}{596 MHz} & \multicolumn{2}{c|}{670 MHz} & \multicolumn{2}{c|}{710 MHz} \\
     \cline{3-16}
     & & Busy & Idle & Busy & Idle & Busy & Idle & Busy & Idle & Busy & Idle & Busy & Idle & Busy & Idle \\
    \hline
       \multirow{3}{11mm}{Boat Racing}  & Intro & 208.5 & 134.7 & 253.7 & 67.25 & 283.2 & 149.9 & 423.6 & 124.5 & 581.6 & 79.85 & 633.2 & 162.2 & 703.2 & 184.2 \\
       \cline{2-16}
        & Still & 232.3 & 145.5 & 358.0 & 57.00 & 402.8 & 153.0 & 573.9 & 92.17 & 576.3 & 150.0 & 813.9 & 145.0 & 866.3 & 205.9 \\
         
         & GPU engy. error (\%) & 232.3 & 145.5 & 358.0 & 57.00 & 402.8 & 153.0 & 573.9 & 92.17 & 576.3 & 150.0 & 813.9 & 145.0 & 866.3 & 205.9 \\
         
         \hline
        \multirow{2}{11mm}{Bricks Breaker}  & Intro & 140.7 & 70.33 & 158.3 & 69.32 & 204.3 & 109.9 & 166.8 & 134.4 & 130.0 & 141.3 & 115.5 & 172.9 & 133.2 & 191.4 \\
       \cline{2-16}
         & Still & 180.8 & 72.69 & 185.5 & 72.63 & 177.7 & 113.0 & 243.6 & 112.8 & 271.3 & 115.6 & 375.7 & 138.1 & 441.2 & 154.7 \\
         \hline
        \multirow{2}{13mm}{Candy Crush Saga}  & Into & 228.0 & 83.88 & 262.5 & 83.97 & 344.4 & 112.7 & 523.1 & 123.3 & 620.7 & 120.0 & 792.6 & 146.2 & 900.4 & 167.2 \\
        \cline{2-16}
         & Tutorial & 226.3 & 92.85 & 228.4 & 89.50 & 330.6 & 117.7 & 445.2 & 134.5 & 519.6 & 137.5 & 660.2 & 160.7 & 855.3 & 171.6 \\
         \hline
    \end{tabular}
    \label{tab:gpumodel_motoz3}
    \vspace{-0.1in}
    }
\end{table*}

% Comparison Table for GPU model with other scenarios Here

We repeated the GPU model derivation for three
popular games apps, Boat Racing, Candy Crush Saga and Bricks breaker.
each running two scenarios, as listed in Table~\ref{tab:app_scenario_description}.
To minimize the variance of CPU power draw, we fixed the CPU frequency at 1.056 GHz,
and ran each app scenario under each GPU frequency for a duration of 60 seconds.

\paragraph{Derived model}
Table~\ref{tab:gpumodel_motoz3} shows the derived GPU power models for varying GPU frequencies for Moto Z3 and Table~\ref{tab:gpumodel_nexus6} for Nexus 6.
%% 5a. Explain per scenario dependent modeling
We make two observations.
(1) The GPU power parameters for the same frequency differ with {\it different apps}; at 710 MHz, the GPU Busy power draw for Boat Racing is 10.6\% lower than for Candy Crush Saga on Moto Z3.
(2) The GPU parameters for the same frequency even differ for {\it different scenarios} 
of the same app; for Boat Racing, the GPU Busy power draw
for the Intro scenario is 10.2\% and 29.7\% lower than for the Still scenario
at 257 MHz and 414 MHz, respectively.

To show the importance of app-specific power modeling, we calculated the error
in estimating the total GPU energy drain in the second scenario of each app 
if using the GPU model derived using the first scenario of each app.
Table~\ref{tab:gpumodel_motoz3} shows that the error ranges between 
??\%--??\%, ??\%--??\%, and ??\%--??\% for the second scenarios of the three apps.


%% 5b. Explain the reasons for the observation
The above dependence of the GPU power draw on app usage can be attributed  to two main reasons.
First, the GPU has a large number of mini-cores, but the utilization metric available to the OS only captures the temporal utilization but not the spatial utilization, \ie the percentage of mini-cores that were active.
Different spatial utilization may drive the GPU into different power state variations that have the same temporal utilization but different power draw.

Second, rendering different frames in different scenarios of the same app or different apps may result in different memory operation mix, and hence using a single CPU model in estimating the CPU power draw portion to be subtracted from the total phone power 
may result in errors in the GPU power draw  estimation, as we see in \S\ref{sec:primer_cpu}. 
Such error propagation happens in TPMD which models one component at a time
and often relies on the models for prior components to estimate the "ground truth" in modeling  the next component. 