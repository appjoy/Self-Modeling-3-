\begin{abstract}

Power modeling of mobile device components is foundational to mobile app energy
	drain analysis such as for energy profiling and energy debugging, and to
	mobile OS design such as %runtime power management 
	for battery-drain accounting and energy-aware
	scheduling.  Traditional power model derivation (TPMD) is an offline process
	based on microbenchmarks
% to drive each mobile device component into each possible power state to
	% record its power draw (from a power monitor) 
	but it is a time consuming process, taking several weeks of manual effort
	to derive a phone's power model, and is often not comprehensive-- it misses
	the dependence of component power behavior on model parameters not considered
	at the modeling time such as battery age and differing mobile network
	behavior in different geographies. Thus, porting mobile OSes, such as
	Android, over to new devices adds an additional overhead for OEMs to provide
	a power model.  \comment{I've seen these to be just copy-pasted from
	other devices and to be wrong in many cases, can we make such a claim, do we
	have some examples? Yes.
	Android power profile is stored in power\_profile.xml~\cite{androidpowerprofile}.
	We found out that Pixel 4 and Pixel 2 share the same settings
	whereas the CPU are from different generations of SoC.}
	Similarly, due to the tedious nature of
	TPMD, energy profilers have not been able to become plug-and-play.
	Commercial mobile energy profilers either support a handful of devices or show
	very coarse results. Further, since power models generated from TPMD are not comprehensive, mobile
	energy profilers can show inaccurate results that can confuse developers.

	% app usage, the mobile device, and environment such as the battery age.}
% or device temperature. 
In contrast, self-constructive power model derivation (SPMD)
	promises to overcome these shortcomings of TPMD by collecting 
% each actual power state that each device component ran under, the
	% utilization, and 
component power state and utilization, total device energy drain during an app
	run, and deriving the power model parameters by regression using a system of
	linear equations.  
%	\st{Though first proposed almost a decade ago, SPMD has
%	only been explored so far for energy drain estimation with a finer resolution
%	(e.g. 10ms) compared to the device power sensor readings (e.g. 1s). }
	Although first proposed almost a decade ago, SPMD has only been
	explored so far for enhancing the resolution for total energy drain.
	In this paper, we report our experience of applying SPMD in deriving
	{\it component-wise} power models for modeling 
	%component-wise 
	energy consumption on 3 modern
	smartphones. 
	% Accurate component-wise power models are necessary for the 
	% design of energy profilers and energy-aware schedulers. In this paper, we explore
	% the feasibility of SPMD on modern smartphones. We find that the time window granularity 
	% used for setting up the systems of equations plays a major roles in the success of SPMD
	% and show promising results of applying SPMD to quickly generate comprehensive component
	% power models for three modern devices.
	%, which potentially provide more valuable information for
	%real-world application (e.g., diagnosing power draw , battery scheduling)
	%than just enhancing resolution of energy drain estimation. 
	%By our experiment,
	Overall, we find that the measurement noise and lack of equation diversity 
	greatly impact the modeling results of
	SPMD, especially when sampling frequency is low. When using higher sampling
	frequency, we get relatively better modeling results, but is still not good
	enough to be used in practice.
	% , which suggests that the utilization of SPMD on
	% modern smartphones has a high requirement on the quality of measuring and
	% sampling.
	

\st{
In this paper, we experimentally study 
the feasibility of SPMD in deriving component-wise power models 
for modern smartphones. Our study shows that SPMD faces a 
fundamental dilemma: it can only be used for the particular 
app scenario as even different scenarios of the same app result 
in different power models; on the other hand, when 
restricted to a particular app scenario, the system of 
equations lacks diversity when created at the per-second granularity, 
preventing the regression solver from generating meaningful model parameters.  
%  Finally, we show
% zooming into the millisecond scale in setting up equations can overcome the diversity problem 
% but the equation can be dominated by measurement noise, and further SPMD would not scale
% as a 1-minute app run would require solving thousands of systems of equations. 
Our study suggests that it is extremely challenging to derive component-wise power model parameters for modern smartphones using SPMD. }


\if 0
In this paper, we perform an in-depth investigation of the feasibility of SPMD in deriving accurate per-component power models for modern smartphones. 
 % We systematically explore the time-scale for setting up the system of equations, ranging from macro-scale (one equation per second), to micro-scale (two equations per rendering interval to exploit GPU Busy and Idle power states), to nano-scale (16 equations per rendering interval to explore diversity within each rendering interval.)
We find that while conceptually simple, in practice it is 
extreme difficult to create a system of equations that 
exhibit sufficient diversity which poses two challenges 
to the regression solver to generate meaningful per-component 
power model parameters: (1) lack of diversity in phone
component usage which leads to the rank of the equations 
(right-hand-side of equations) to be smaller than the number 
of unknowns (model parameters); (2) noise in energy drain 
readings (left-hand-side of the equations) which results 
in contradicting component usage and energy drain. Our 
experimental results suggest that while SPMD may be useful for
energy modeling which only requires a good fitting of the two 
sides of the equations, it remain not practical for deriving 
per-component power model parameters for use in mobile app 
energy drain analysis such as energy profiling and energy debugging. 

%   We believe that our  ndings can encourage further research and standardization e orts towards higher utilization of commercialized CIoT network infrastructures.
\fi

\end{abstract}
