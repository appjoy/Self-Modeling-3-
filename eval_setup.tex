\section{Experimental Setup}
\label{sec:etup}


%% 2. Explain the device used

% We carried our SPMD on three phones with fairly diverse hardware.
% Moto Z3 is a representative of recent phone.
% It has a Kryo Qualcomm SOC~\cite{snapdragon835} based on the big.LITTLE architecture
% with 4 LITTLE cores (22 frequencies) and 4 big cores (31 frequencies), and
% an Adreno 540 GPU which can run at 7 different frequencies.
 
% In contrast, the Nexus 6 was a representative of the generation before
% the big.LITTLE architecture, with a Krait Qualcomm SOC~\cite{snapdragon805} that
% has 4 cores which can independently operate at 18 different frequencies, and
% an Adreno 420 GPU  which can run at 5 different frequencies.

% In our experiments, we simplified the SPMD modeling task by turning off the LITTLE cores on Moto Z3
% % for all our experiments, hence all our results are based on big cores.
% which reduces the number of parameters in the modeling equations.
% As we will see below, even with this simplification, making SPMD to work is already difficult. 

% Since we focus our SPMD experiments on two phone components, the CPU and GPU,
% we used a set of {three} popular game apps which are known to predominantly
% use only the CPU and GPU. The apps have diverse GPU and CPU utilization. 
% For example, the average GPU utilization is 38.0\% for Pottery,
% 25.92\% for Candy Crush Saga and 16.0\% for Bricks Breaker on Moto Z3.

We carried our SPMD on three representative phones from three generations,
Pixel 2, MotoZ3, and Pixel 4 released in 2017, 2018, and 2019, respectively.
Pixel 2 and Moto Z3 have a Kryo Qualcomm SOC Snapdragon 835 each~\cite{snapdragon835} and Pixel
4 has a Snapdragon 855~\cite{snapdragon855}. Both SOCs are based on the
big.LITTLE architecture.  
% Pixel 2 and Moto Z3 
Snapdragon 835 has a CPU with 22 LITTLE-core and 31 big-core frequencies, 
and an Adreno 540 GPU which can run at 7 different frequencies.
Snapdragon 855 has a CPU with 18 LITTLE-core and 20 big-core frequencies,  
and an Adreno 640 GPU which can run at 5 different frequencies.

For our experiments, we simplified the SPMD modeling task by turning
off the LITTLE cores \dcomment{ and 2 big cores}
on the phones % Moto Z3
% for all our experiments, hence all our results are based on big cores.
which reduces the number of parameters in the modeling equations.
As we will see below, even with this simplification, making SPMD to work is difficult. 
Since we focus our SPMD experiments on two of the phone components, the CPU and GPU,
we use a set of {three} popular game apps, shown in Table~\ref{tab:app_scenario_description},
that are known to be predominantly
using only the CPU and GPU. We found that the apps have diverse GPU and CPU utilization. 
% For example, the average GPU utilization is 38.0\% for Mini Golf 3D,
% 25.92\% for Candy Crush Saga and 16.0\% for Bricks Breaker on Moto Z3.
% \comment{Utilisation not final yet???}
For example, the average GPU utilization is 49.57\% for Mini Golf 3D Still,
22.32\% for Candy Crush Saga Still and 12.20\% for Bricks Breaker Still on Moto Z3.
We further consider two scenarios from each of these apps,
as listed in Table~\ref{tab:app_scenario_description}.

\begin{table}[tp]
    \centering
    \caption{App scenario description.}
    \vspace{-0.1in}
    {\small
        \begin{tabular}{|p{15mm}|p{12mm}|p{46mm}|}
        \hline
        App & Scenario & Description \\
        \hline
        \hline
             \multirow{2}{15mm}{Bricks Breaker} & Menu & Menu page\\
             \cline{2-3}
             & Still & Game running without any input \\
             \hline
             \multirow{2}{15mm}{Candy Crush Saga} & Menu & Menu Page\\
             \cline{2-3}
             & Still &  Game running without any input \\
             \hline
             \multirow{2}{15mm}{Mini Golf 3D} & Menu & Menu Page \\
             \cline{2-3}
             & Still &  Game running without any input \\
            \hline
        \end{tabular}
    }
    \label{tab:app_scenario_description}
    \vspace{-0.2in}
\end{table}

We generated a 250-second run for each of the app scenarios with the
phone attached to the power monitor.  For the three phones Pixel 2,
Moto Z3 and Pixel 4, we kept the screen dark and removed its constant
power draw of 43mA, 57mA and 78mA, respectively, from the power monitor
measurement in setting up the equations.  For each scenario, we left the
app running without any user input.

% \paragraph{Aligning power trigger trace with power monitor readings}
% When readings from the Monsoon external power monitor, either as ground truth in model accuracy calculation or as the left-hand-side (LHS) value of the modeling equations, 


% The results we observed from our experiment can be easily extended by including corresponding triggers for the case while all the cores are running .

% Finally, we found that the findings for the three phones are  very similar.
% We thus moved all the results for Nexus 6 
% to~\cite{longversion} due to page limit.
