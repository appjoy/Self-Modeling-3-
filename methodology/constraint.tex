\subsection{Adding Constraints to the Solver}
\label{subsec:constraint}

As we will see in \S\ref{sec:modelling_macro}, because the regression
solver tries to minimize the fitting error and is oblivious to the
semantics of the unknowns, it may output solutions that minimize the
fitting error but violate basic properties of the power model
parameter values, such as non-negativity {\color{blue}(all coefficients should not be negative)} and monotonicity {\color{blue}(coefficients of higher frequency should be larger)}.  To overcome
this, in addition to the basic unconstrained regression solver, {\color{blue}we also experimented two constrained regression solver} based on the domain knowledge of power models, as shown in
Table~\ref{tab:constrained}.

\begin{table}[tp]
%\questionaj{These constraints must also constrain GPU busy and idle.}
{\small
    \centering
    \caption{Constraints added to the regression solver.}
    \vspace{-0.1in}
    \begin{tabular}{|p{25mm}|p{52mm}|}
    \hline
         Constrained SPMD  & \multicolumn{1}{c|}{Description} \\
         \hline
\hline
         Unconstrained      & No constraints are applied. \\
\hline
         Constrained        & Model parameters should not decrease with increasing frequencies. GPU idle power is less than GPU busy power. \\
\hline
         Freq-constrained   & CPU model parameters are modeled as a polynomial of frequency. \\
         \hline
    \end{tabular}
    \label{tab:constrained}
    \vspace{-0.1in}
}
\end{table}


\paragraph{Constraint 1: positivity and monotonicity.}
Without any constraint, we found the regression solver can output
CPU/GPU power values that are negative or decreasing when the
frequency increases.  To prevent the solver from outputing such
solutions, we added three constraints to the regression solver: (1) all
model parameters should be positive; (2) the model parameters with
increasing frequencies for each component should be non-decreasing;
(3) the GPU Idle power should be lower than the GPU Busy power.  We
denote this version of SPMD as "Constrained SPMD".


\paragraph{Constraint 2: modeling CPU power as a function of frequency.}
We found even with Constraint 1, the CPU power parameters with varying
frequencies output by the solver often remain the same.  To make the
CPU model parameters output more consistent with the reality, we
exploited another domain knowledge about the CPU power draw, that it
increases as a low-order polynomial of the operating
frequency~\cite{armdvfs,rizvandi2011some}.
% Since the specific polynomial function may vary for different phones, \eg early phones only performed % frequency scaling (FS) while newer phones perform dynamic voltage and frequency scaling (DVFS), we first used a simple CPU benchmark to find the order of the polynomial for each phone used in our experiments. 
%
\if 0
\begin{figure}[tp]
    \centering
%    \vspace{-0.1in}
    \includegraphics[width=0.80\columnwidth]{figures/cpu_characteristics.png}
    \caption{Moto Z3 big core CPU power draw grows with frequency 
    when running the arithmetic-intensive benchmark.}
    \label{fig:cpu_characteristic}
    \vspace{-0.1in}
\end{figure}
\fi
%
\cut{ We ran a microbenchmark that performs arithmetic-intensive or memory-intensive operations for 7 seconds
on 1, 2 and 4 big cores, for each big core frequency, and fit the measured phone power 
with a polynomial function in frequency.
%\begin{equation}
%     P_{CPU} = p^c_{base}+ \sum_i p^c_i({f_k}) = p_{base}+\sum_i  a*f_k^{n}
% \end{equation}
% Figure~\ref{fig:cpu_characteristic} shows the measured power draw and curve fitting on Moto Z3.
}
Our CPU microbenchmark results show  that 
% the fitted exponent $n$ for 1, 2 and 4 cores are 3.01, 2.97 and 3.01, respectively.
the per-core power grows as the third-power of the CPU
frequency on all three phones, 
\ie $p_i^c(f_k) = a\cdot f_k^3$. 
Thus we replace all the CPU power parameters with
third-power of the CPU frequency in the SPMD model equations.
%model the Moto Z3 big core power as a third-order polynomial of the CPU frequency in the model equations of SPMD.
%
Doing so has two benefits: (1) it reduces the number of CPU power
parameters from $K$, the number of CPU frequencies, to 2, the base power $p_{\text{base}}$
and coefficient $a$, which reduces the number of equations needed for
the solver; (2) it constraints the CPU power to be not only
monotonically increasing with its frequency, but also consistent with its physical power behavior.
