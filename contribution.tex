\subsection{Our Contribution}
\label{subsec:contri}


In this work, we study the feasibility of SPMD to derive 
accurate {\it component-wise power models} for modern smartphones.
In order to present an in-depth analysis within  the  page limit, 
we focus our feasibility study on two of the most power-hungry 
phone components, the CPU and GPU, on representative phones from three generations, 
Pixel 2 (released in 2017), Moto Z3 (2018), and Pixel 4 (2019).

Our key findings are summarized as follows:
\begin{itemize}
\item 
% 	To gain insight into the suitable granularity of app run for applying 
% an instance of SPMD, \ie the app run interval for which a system of equations
% is set up and solved and the derived model can be used for,
% we start with a measurement study of the power model dependence on
% app usage. 
	Our results show that both the CPU and the GPU of the 
three phones exhibit significant power state variations, \ie their 
power draw for the same power state (for same CPU frequency and same 
GPU frequency and state) can vary up to  13.5\% and 29.7\% for CPU and GPU, respectively.
% even when running different scenarios of the same apps. 
This result suggests that (1) any power model derived via 
TPMD offline based on a given microbenchmark 
can be inaccurate when used to model the power draw of 
different apps or different app scenarios;
(2) different scenarios of the same app should not be used in 
the same instance of SPMD. 
    
% \cut{
% \item The built-in power sensor readings available in 
% modern smartphones can have as much as 17.99\% average 
% error compared to an external high-resolution power monitor 
% and hence are not accurate enough to be used in SPMD. Thus we focus 
% our study using the power monitor. 
% }
%

{\color{blue}\item We use statistical tools such as F-test and $R^2$-measurement \st{to} for examining the SPMD modeling results from linear regression solutions. Those statistical testing results show that the modeling results fail to explain the variability of total power consumption.\dcomment{ \st{The main reason is that}} The key reason is that the CPU/GPU usage for the app scenarios such 
as game apps are highly repetitive at the per-second time scale. This repetition directly makes some of the singular values of the design matrix in the linear regression close to zero. As a result, the regression result becomes very sensitive to even small amount of noise}\footnote{{\color{blue}Intuitively, consider a linear problem $\mathbf{Y}=\mathbf{X}\beta+\epsilon$. The amount of impact of the noise $\epsilon$ to the solution of $\beta$ is about $\|\mathbf{X}^{\dagger}\epsilon\|_2$, where $\mathbf{X}^{\dagger}$ denotes the pseudo-inverse of the design matrix $\mathbf{X}$. Because the noise vector $\epsilon$ is random, so $\epsilon$ may have a component vector $\epsilon_m$ that is in the direction of the smallest singular vector of $\mathbf{X}$. If the smallest singular value of $\mathbf{X}$ denoted by $\min\text{eig}(\mathbf{X})$ is close to zero, then $\|\mathbf{X}^{\dagger}\epsilon_m\|_2=\|\epsilon_m\|_2/\min\text{eig}(\mathbf{X})$ will be extremely large, and thus makes the impact of noise $\|\mathbf{X}^{\dagger}\epsilon\|_2$ very large.}}.

\st{
% \item 
However, when limiting SPMD within each of the app scenarios, creating 
one equation at the per-second time scale 
cannot create a system of equations with enough diversity for 
the regression solver to output any meaningful power model parameters.
%
\cut{Despite adding many constraints to the regression solver,
the GPU Idle power is as high as the GPU Busy power. }
The key reason is that the CPU/GPU usage for the app scenarios such 
as game apps are highly repetitive; 
the systems of equations often have a single large singular value.  }
\item 
{\color{blue}In order to alleviate this problem of close-to-zero singular values caused by repetitive CPU/GPU utilization level}, we zoom into a smaller time scale, by setting each 
equation {\color{blue}used in the linear regression} at the millisecond time scale
to explicitly exploit the {\color{blue} variation of the GPU utilization}. Our results show that doing 
so can 
make the regression solver to  output sometimes meaningful model parameters, {\color{blue}but not always. We find that sometimes measurement noise is still large enough to alter the solutions quite significantly.}
\st{Further, SPMD at such a fine granularity would not scale well as a 1-minute app run would require solving thousands of systems of equations. }
\if 0
Our experiments show that the system still does not have enough diversity  
    at either the micro-scale, \ie with 2 equations in each rendering interval to exploit the GPU state diversity (Busy/Idle), or the nano-scale, \ie with 16 equations in each 16.7ms rendering interval, for two reasons: (1) the component usage lacks diversity,
    and (2) the noise in the coefficient measurements overshadows the diversity of utilization coefficients in the equations. 
\fi
\end{itemize}
    
% \st{These findings from our exploration strongly suggest that while SPMD may be useful for energy modeling which only requires a good least-square fitting of the two sides of the energy equations~\cite{sesame:2011}, it remains extremely challenging to derive component-wise power model parameters .}
{\color{blue}These findings from our exploration conveys some useful guidance for implementing SPMD on modern smartphones: 1) repetitive CPU/GPU utilization level makes SPMD harder to output reasonable modeling results; 2) finer resolution may help; 3) more accurate measurement with low noise is critical for reliable SPMD modeling results.}

