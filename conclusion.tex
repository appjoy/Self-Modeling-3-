\section{Conclusion}
\label{sec:conc}

In this paper, we performed an in-depth study of the feasibility of
self-constructive per-component power model derivation (SPMD) on
modern smartphones. {\color{blue}Although our experiment results show that SMPD cannot always produce reasonable modeling results due the existence of noise, we show a detailed process of implementing SMPD on modern smartphones. Meanwhile, we also provide useful guidance to improve the performance of SMPD by choosing suitable granularity to exploit the variation of the utilization level of CPU/GPU.}
\st{Using three representative phones, we identified the
equation diversity as the primary challenge of SPMD and explored
diverse time-scales in trying to create equations that exhibit high
diversity.  Our experiments and analysis have shown that it is
extremely difficult to create a system of equations with sufficient diversity for
the regression solver to generate meaningful per-component power model
parameters.  Our feasibility study debunked a myth about the
applicability of SPMD at the per-component level of modern smartphones.}